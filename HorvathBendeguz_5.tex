\documentclass[12pt]{article}
 \usepackage[margin=1in]{geometry} 
\usepackage{amsmath}
\usepackage[utf8]{inputenc}
%\usepackage[ utf8 ]{inputenc}
\def\doubleunderline#1{\underline{\underline{#1}}}
\newcommand{\angstrom}{\text{\normalfont\AA}}
\usepackage[magyar]{babel}
\newcommand{\N}{\mathbb{N}}
\newcommand{\Z}{\mathbb{Z}}
\usepackage{t1enc}
\usepackage{placeins}
\usepackage{caption}
\usepackage{textcomp}
\usepackage[utf8]{inputenc}
\usepackage[T1]{fontenc}
\title{Véletlen fizikai folyamatok, ötödik házi feladat}
\author{Horváth Bendegúz}

\begin{document}
\maketitle
\section*{1. feladat}
\subsection*{A feladat szövege}
Vizsgáljuk az előadáson tárgyalt 2 Ising spinből álló rendszer relaxációjának problémáját! Az előadáson megkaptuk a rendszer dinamikai mátrixát, s meghatároztuk a sajátvektorokat és a megfelelő sajátértékeket (A számolás megtalálható a kurzus honlapján ”Ising spinek dinamikája” cím alatt is).\\
(i) Ismerve az összes sajátvektort és sajátértékeket, határozzuk meg milyen valószínűséggel van a rendszer t időpontban az $s_1 = {-1}$, $s_2 = +1$ állapotban, ha a kezdeti állapot $s_1 = +1$,$s_2 =-1$ volt.\\
(ii) Számítsuk ki a rendszer átlagos mágnesezetségének $[M(t) = \langle(s_1+s_2)\rangle] $időfejlődésést, ha a kezdeti állapotban az $s_1 = 1$, $ s_2 = 1$ állapot és az $s_1 = -1$, $s_2 = +1$ állapot is 1/2 valószínűséggel van jelen.
\subsection*{A feladat megoldása}
(i) A sajátvektorokat és a sajátértékeket ismerjük, ezzel a tudással egy állapot valószínűségét így számolhatjuk ki:
$$\vec{P}(t) = P^{(e)}(t)+\sum^4_{i = 2}a_ie^{\lambda_it}\vec{P^{(i)}}$$
az $a_i$ együtthatókat a kezdeti feltételekből számolhatjuk, ami esetünkben $(\uparrow\downarrow)$ állapot. A sajátvektorok a következőek:
\[
\vec{P}^{(e)} = \frac{1}{2(e^{K}+e^{-K})}
\begin{pmatrix}
	e^{K}\\
	e^{-K}\\
	e^{-K}\\
	e^{K}
\end{pmatrix}
\hspace{1cm}
\vec{P}^{(2)}=
  \begin{pmatrix}
	1\\
	-1\\
	-1\\
	1
\end{pmatrix}
\hspace{1cm}
\vec{P}^{(3)} =
\begin{pmatrix}
	1\\
	0\\
	0\\
	-1
\end{pmatrix}
\hspace{1cm}
\vec{P}^{(4)}
\begin{pmatrix}
	0\\
	1\\
	-1\\
	0
\end{pmatrix}
\]
A sajátértékek pedig $\lambda_1 = 0$, $\lambda_2 = -2(1 + e^{-2K})$, $\lambda_3 = -2e^{-K}$ és $\lambda_4 = -2$. Itt a $K$ értékek az előadáson használt jelölés, $K = \beta J$, $\beta = 1/k_bT$.
A kezdeti feltétel:
\[
\vec{P}(t = 0) = 
\begin{pmatrix}
	0\\
	0\\
	1\\
	0
\end{pmatrix}
= \vec{P}^{(e)} + \sum^{4}_{i = 2}a_i\vec{P}^{i}.
\]
Ezeből a kövekező egyenleteket kapjuk az $a_i$ együtthatókra:
\begin{equation}
\frac{e^K}{2(e^K+e^{-K})}+a_2+a_3 = 0
\end{equation}
\begin{equation}
\frac{e^{-K}}{2(e^K+e^{-K})}-a_2+a_4 = 0
\end{equation}
\begin{equation}
\frac{e^{-K}}{2(e^K+e^{-K})}-a_2-a_4 = 1
\end{equation}
\begin{equation}
\frac{e^K}{2(e^K+e^{-K})}+a_2-a_3 = 0
\end{equation}
Kivonva az (1)-ből a (4)-est, $a_3 = 0$. Ezt behelyettesítve az (1)-esbe, $a_2 = -\frac{e^K}{2(e^K+e^{-K})}$. A (2)-ből a (3)-ast kivonva $ a_4 = -0.5$.\\
Most összegezzük fel a végeredményt, de mi a $\vec{P}$-nek csak a második komponensére vagyunk kíváncsiak:
$$\vec{P}(\downarrow\uparrow, t) = \frac{e^{-K}}{2(e^K+e^{-K})} + \frac{e^{-K}}{2(e^K+e^{-K})}\cdot e^{-2(1 + e^{-2K})t}-\frac{1}{2}e^{-2t}$$
Így ez lett a keresett valószínűség.
\\\\
(ii) Hasonló módon kell csinálni a mágnesezettséggel is:
$$M(t) = \langle( s_1+s_2)\rangle = \sum_{s_1=\pm1}\sum_{s_2=\pm1}(s_1+s_2)P(s_1, s_2, t)$$
A kezdeti feltétel most $\vec{P}(t = 0) = (1/2, 0,0, 1/2)$. Erre a következő egyenletek adódnak:
\begin{equation}
\frac{e^K}{2(e^K+e^{-K})}+a_2+a_3 = \frac{1}{2}
\end{equation}
\begin{equation}
\frac{e^{-K}}{2(e^K+e^{-K})}-a_2+a_4 = 0
\end{equation}
\begin{equation}
\frac{e^{-K}}{2(e^K+e^{-K})}-a_2-a_4 = 0
\end{equation}
\begin{equation}
\frac{e^K}{2(e^K+e^{-K})}+a_2-a_3 = \frac{1}{2}
\end{equation}
Hasonlóan az előzőekhez, (5)-ből kivonva (6)-ost kapjuk, hogy $a_3 = \frac{1}{2}$. visszahelyettesítve az (5)-be, $a_2=-\frac{e^K}{2(e^K+e^{-K})} $, a (6)-osból kivonva a (7)-est $a_4 = 0$-t kapunk. Így megkaphatjuk a $\vec{P}(t)$-t:
\[
\vec{P}(t) = \frac{1}{2(e^K+e^{-K})}
\begin{pmatrix}
e^{K}\\
	e^{-K}\\
	e^{-K}\\
	e^{K}
\end{pmatrix} 
+ \frac{e^{-K}}{2(e^K+e^{-K})}\cdot e^{-2(1 + e^{-2K})t}
\begin{pmatrix}
	1\\
	-1\\
	-1\\
	1
\end{pmatrix}
+ \frac{1}{2}e^{-2e^{-K}t}
\begin{pmatrix}
	1\\
	0\\
	0\\
	-1
\end{pmatrix}
\]
Az $M(t)$-re felírt egyenletben az $s_1+s_2 = 0$ résznél kiesnek az $\uparrow\downarrow$ és $\downarrow\uparrow$ állapotok, megmaradnak viszont az $\uparrow\uparrow$ és $\downarrow\downarrow$ állapotok. Így a mágnesezettség:
$$ M(t) = 2\cdot \big (\frac{1}{2(e^K+e^{-K})} e^K + \frac{e^{-K}}{2(e^K+e^{-K})}\cdot e^{-2(1 + e^{-2K})t} + \frac{1}{2}e^{-2e^{-K}t} \big ) - ...$$ 
a $\vec{P}(t)$-re kapott összegből az első kettő tagnak a vektorainak az első és negyedik komponens ugyanaz, így azok kiejtik egymást, marad az utolsó tag:
$$M(t) = 2\cdot\frac{1}{2}e^{-2e^{-K}t}- (-2)\cdot\frac{1}{2}e^{-2e^{-K}t} = 2\cdot e^{-2e^{-K}t} $$

\newpage
\section*{2. feladat}
\subsection*{A feladat szövege}
Egy háromszögön egy részecske ugrál a szomszédos csúcsok között. Az ugás rátája $w_1$
az óramutatójárásával egy irányban és $w_2$ az ellenkezőirányban. Feladatok:\\\\
Írjuk fel a master egyenletet az i-edik csúcsban tartózkodás valószíűségére!\\ 
Határozzuk meg a stacionáris megoldást!\\
Határozzuk meg a rendszer rrelaxációs idejét  (előszőr próbáljuk megbecsülni az értékét)!

\subsection*{A feladat megoldása}
$i  \epsilon  \{1, 2,3 \}$, így az egyenletrendszerünk a következő:
$$\partial_t P_1 = -(w_1 + w_2)P_1 + w_2P_2 + w_1P_3$$
$$\partial_t P_2 = -(w_1 + w_2)P_2 + w_2P_3 + w_1P_1$$
$$\partial_t P_3 = -(w_1 + w_2)P_3 + w_2P_1 + w_1P_2$$\\
A következő átmeneti mátrixot használom majd:
\[
\doubleunderline\alpha=
  \begin{pmatrix}
    -(w_1+w_2) & w_2 & w_1 \\
    w_1&-(w_1+w2)&w_2\\
    w_2&w_1&-(w_1 + w_2)
    
  \end{pmatrix}
\] Így a következő sajátértékegyenletet írhatjuk fel:
$$\partial _t\vec{P} =\doubleunderline\alpha  \vec{P},$$
$\vec{P} = (P_1, P_2, P_3)$ jelöléssel. Stacionárius állapotban az egyenlet baloldala nulla, a $0$-ás sajátértékhez tartozó sajátvektor a megoldásunk. 
$$0\vec{P} = \doubleunderline\alpha_0 \vec{P}.$$
A homogén, lineáris egyenletrendszer túlhatározott, így kell mellékfeltételként: $\sum^3_{i = 1}P_i = 1$ és $P_i> 0.$ A következő vektor kielégíti az egyenleteket:
\[
\vec{P}=
  \begin{pmatrix}
    \frac{1}{3(w_1+w_2)}\\
   \frac{1}{3(w_1+w_2)}\\
    \frac{1}{3(w_1+w_2)}
    
  \end{pmatrix}
\]
A rendszer szimmetrikussága miatt nem meglepő a stacionárius állapotra kapott eredmény.($\vec{P} = 1/3(1,1,1))$, mert $w_1 + w_2 = 1$. Relaxációs idő számolásához ismét felírhatjuk a sajátérték egyenletet, de most általános esetben oldjuk meg:
$$\partial _t \vec{P_\alpha} = \lambda_\alpha \vec{P_\alpha},$$ 
ahol $\lambda_\alpha $ az $\alpha$-adik sajátértéke. A megoldást kereshetjük a következő alakban:
$$\vec{P}_\alpha (t) = a_\alpha e^{-\lambda \frac{t}{\tau}}\vec{P_\alpha}$$
A relaxációs idő pedig $\tau_{relax} = \frac{\tau}{|\lambda_{min}|}, $ ha $\tau$ időközönként van átmenet. Megoldva a sajátértékegyenletet, a következő egyenletet kapjuk $\lambda$-ra:
$$\lambda_{1,2} = \frac{-3(w_1+w_2)\pm\sqrt{12\cdot w_1w_2-3(w_1+w_2)^2}}{2}$$
$$\lambda_{min} =\frac{-3(w_1+w_2)-\sqrt{12w_1w_2-3(w_1+w_2)^2)}}{2} $$
És így:
$$\tau_{relax} = \frac{2\tau}{3(w_1+w_2)+\sqrt{12w_1w_2-3(w_1+w_2)^2}}$$.
Esetünkben két felé léphet a részecske, $w_1 + w_2 = 1$, így kicsitt szebben is lehet írni:
$$\tau_{relax } \frac{2\tau}{3+\sqrt{12w_1w_2-3}}$$

\newpage
\section*{3. feladat}
\subsection*{A feladat szövege}
Meredek hegyoldalban függőegesen $l$ távolságra vannak a kapaszkodók. A hegymászó $w$ rátával lép felfelé, s $w_0$ annak a rátája, hogy lecsúszik egy szintet, s onnan folytatja a mászást.
Feladatok:\\
(i) Irjuk fel az egyenletet, amely meghatározza, hogy a hegymázó milyen $P_n$ valószínűségel van $nl$ magasságban!\\
(ii) Használjuk a generátorfüggvény formalizmust a stacionárius eloszlás kiszámítására!\\
 (iii) Határozzuk meg, hogy átlagosan milyen magasra jut a hegymászó!
\subsection*{A feladat megoldása}
A master egyenlet az $nl$ magasságban tartózkodás valószínűsége  a következőképpen néz ki:
$$ \partial_tP_n = -(w + w_0)P_n + w_0P_{n+1} + wP_{n-1},$$
$n = 0$ esetben viszont:
$$\partial_t P_0 = -wP_0 + w_0P_1$$
Legyen
$$q = \frac{w}{w_0} \text{ és } \tau = \frac{1}{w_0} .$$
Így az egyenletek:
$$\tau \partial_t P_n = -(q+1)P_n + P_{n+1} + qP_{n-1}$$
$$ \tau \partial_t P_0 = -qP_0 + P_1.$$
Alkalmazva a generátorfüggvény formalizmusát:
$$\tau \partial _t G(s, t ) = \sum_{n = 0}e^{-sn}P_n = -qP_0 + P_1 + \sum_{n =1}\big [-(1+q)e^{-sn}P_n + qe^{-sn}P_{n-1} + e^{-sn}P_{n+1}\big]$$
A jobboldali tagban a szummában lévő utolsó tagot a következő alakra lehet hozni:
$$\sum_{n =1}e^{-sn}P_{n+1} = e^s\sum_{n = 1}e^{-s(n+1)}P_{n+1}= e^s\sum_{m = 2}e^{-sm}P_m. $$
Az ultolsó előtti tagot:
$$q\sum_{n = 1}e^{-sn}P_{n-1} = qe^{-s}\sum_{n = 1} e^{-s(n-1)}P_{n-1} = qe^{-s}\sum_{m =0}  P_m.$$
Kiegészítve a a $P_0$-s tagokkal a szummás részt átírva generátorfüggvényekre:
$$\tau \partial_tG(s,t) = (1-e^{-s})P_0+ \big [-1-q+qe^{-s}+ e^s\big ]G(s,t)$$
A stacionárius megoldás kiszámításánál az időszerinti parciális derivált 0, így 
$$G_e(s) = \frac{(1-e^s)P_0}{1-e^s+q-qe^-s} = \frac{P_0}{1-qe^{-s}}$$
$G(s = 0) = 1$ feltétel miatt $P_0 = 1-q.$  Milyen magasra jut a hegymászó átlagosan? A várhatóértéket a generátor fügvényekkel a következő módon számolhatjuk:
$$\langle n \rangle = -\partial_s G_e\vline_{s = 0}$$
Így a deriválás után:
$$\langle n\rangle = -\frac{(1-q)qe^{-s}}{(1-qe^{-s})^2}\vline_{s=0} = \frac{q}{1-q} = \frac{\frac{w}{w_0}}{1-\frac{w}{w_0}}= \frac{w}{w_0-w}.$$
ebből az látszik, hogy ahhoz, hogy $\langle n \rangle$ pozitív legyen  $w$-nek kisebbnek kell lennie mint $w_0$, ami nem egyezik a fizikai képpel, felfelé jutási rátának nagyobbnak kéne lennie. Ez valószínűleg abból fakad, hogy egyenúlyi állapotra vonatkozik, a hegymászónak pedig az egyensúlyi állapot a hegy tetejn lenne, oda igyekszik. mászását pedig úgy képzelhetjük el, mint ahogy az előző házifeladatban a részecske beért a nullapontba. Ha a magasságot, amit átlagosan elér n lépés után a következő módon számoljuk:
$$\langle n\cdot l \rangle = l\cdot n (w-w_0),$$
fizikaibb képet kapunk.

\end{document}