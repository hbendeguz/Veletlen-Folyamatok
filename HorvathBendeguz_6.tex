\documentclass[12pt]{article}
 \usepackage[margin=1in]{geometry} 
\usepackage{amsmath}
\usepackage[utf8]{inputenc}
%\usepackage[ utf8 ]{inputenc}
\def\doubleunderline#1{\underline{\underline{#1}}}
\newcommand{\angstrom}{\text{\normalfont\AA}}
\usepackage[magyar]{babel}
\newcommand{\N}{\mathbb{N}}
\newcommand{\Z}{\mathbb{Z}}
\usepackage{t1enc}
\usepackage{placeins}
\usepackage{caption}
\usepackage{textcomp}
\usepackage[utf8]{inputenc}
\usepackage[T1]{fontenc}
\title{Véletlen fizikai folyamatok, hatodik házi feladat}
\author{Horváth Bendegúz}

\begin{document}
\maketitle
\section*{1. feladat}
\subsection*{A feladat szövege}
Lokalizált mágneses momentum ($\mu$) z-tengely irányú B mágneses térben van. A mágneses
momentum ($\mu$) a tér irányában $\pm\mu B$ értéket vehet fel, s ezekben az állapotokban energiája $ \mp\mu_BB$. A mágneses momentum T hőmérsékletű könyezettel van egyensúlyban, s kölcsönhatás eredményeképpen $\mu$ billeg a $-\mu B$ és $+\mu B$ állapotok között $ (-\mu B \leftrightarrow  +\mu B)$.\\
$(i)$ Írjuk fel a folyamat master egyenletét!\\
$(ii)$ Válasszunk olyan átmeneti rátákat, amelyek kielégítik a részletes egyensúly elvét!\\
$(iii)$ Határozzuk meg az átlagos mágneses momentum $\langle \mu (t)\rangle$ időfejlődését, ha kezdetben (t = 0) a mágneses momentum a z-tengely pozitív irányába mutatott.
\subsection*{A feladat mgoldása}
$(i)$ A rendszer két állpot között változik, így ezekre az állapotokra felírva a master egyenlet:
$$\partial _tP_{-\mu}(t) = -w_0P_{-\mu}(t) + wP_{mu}(t)$$
$$\partial_t P_{\mu}(t)= w_0P_{-\mu}(t) - w P_{\mu}(t)$$
Az egyenletekben $P_{\pm\mu}(t)$ a $\pm\mu$ állapotban tartózkodás valószínűségét jelöli. \\
(ii) Először próbáljuk meg megoldani a sajátértékegyenletet:
\[
\partial_t
\begin{pmatrix}
P_{-\mu}(t)\\
p_{+\mu}(t)
\end{pmatrix}=
\begin{pmatrix}
-w_0& w\\
w_0 & -w
\end{pmatrix}
\begin{pmatrix}
P_{-\mu}(t)\\
p_{+\mu}(t)
\end{pmatrix}
\]

Az Ising-spinrendszerhez hasonlóan, a  $P^{(e)}_{\pm\mu} = \frac{1}{Z}e^{-\beta E(\mp \mu B)}$. A sajátértékek a számolás után a következőknek adódnak: $\lambda_1 = 0$, ez az egyensúlyi állapothoz tarozik, és $\lambda_2 = -(w_0 + w)$. Ezek után a részletes egyensúlyi elvét kielégítő átmeneti rátákat kell meghatározni. Legyen 
$$\frac{w_{n, n'}}{w_{n', n}} = e^{-\beta(E_{n'}-E_{n})}  $$
Ha $\Delta E = (E_{n'}-E_{n})> 0$, akkor az átmeneti ráta
$$w_{n',n} =\frac{1}{\tau}e^{-\beta(E_{n'}-E_{n})} .$$
A $\Delta E = \pm 2\mu_B B$ és pozitív, ha felfelé álló állapotból lefelé álló állapotba ugrott a mágneses momentum. Ha $\Delta E < 0$, akkor $w_{n',n} = \frac{1}{\tau}$, $\tau$ az urgások közti idő. Így 
$$w_0 = \frac{1}{\tau}$$
$$w= \frac{1}{\tau}e^{-\beta 2 \mu_B B}$$
megkaptuk az átmeneti rátákat.\\
(iii) Az előző feladatrészben kiszámoltuk a sajátértékeket, így az állapotban tartózkodás valószínűségét leíró vektort a következő módon kaphatjuk meg:
$$\vec{P}(t) = \vec{P}^{(e)} + e^{\lambda_2 t} a_2 \vec{P}_2$$
Az $a_2$ együtthatót a kezdeti feltételből határozhatjuk meg, ami esetünkben
$$\vec{P}(0) = (0, 1)^T = \vec{P}^{(e)} + a_2\vec{P}_2.$$
$\vec{P}_2 = (1, -1)^T$ sajátvektor, így behelyettesítve az egyenletbe
$$a_2 = -\frac{1}{Z}e^{-\beta \mu_B B}$$
Tehát
\[
 \vec{P}(t) = \frac{1}{Z}
\begin{pmatrix}
e^{-\beta \mu_B B}\\
e^{+\beta \mu_B B}
\end{pmatrix}
-\frac{1}{Z}e^{-\beta \mu_B B} e^{-\frac{1+\frac{1}{\tau}e^{-\beta 2 \mu_B B}}{\tau}t}
\begin{pmatrix}
1\\
-1
\end{pmatrix}
\]
Nekünk viszont a feladat teljes megoldásához a mágnesezettség várható értéke kell, amit a következő módon kaphtaunk meg:
$$\langle \mu(t)\rangle = (-1, 1)^T \vec{P}(t).$$
$$\langle \mu(t)\rangle = \frac{1}{Z}\Big [ -e^{-\beta \mu_B B}+2e^{-\beta \mu_B B}\cdot  e^{-\frac{1+\frac{1}{\tau}e^{-\beta 2 \mu_B B}}{\tau}t} +e^{\beta \mu_B B} \Big ]$$





\newpage
\section*{2. feladat}
\subsection*{A feladat szövege}
Legyen egy egész értékeket felvevő stochasztikus vátozó, n, momentum-generátor fuüggvénye G(s). A normalizációból következik, hogy G(0) = 1, s n momentumai G deriváltján keresztül kifejezhetők:
$$\langle n\rangle = - \frac{dG(s)}{ds} \Big |_{\substack{ s=0 }}\hspace{0.7cm}, ...,\hspace{0.7cm} \langle n^k \rangle = (-1)^k\frac{d^kG(s)}{ds^k} \Big |_{\substack{ s=0 }}$$
A kumuláns generátor függvény a momentum-generáor függvény logaritmusa,
$$ \Phi (s) = \ln{G(s)}$$
s a kumulánsokat a következőképpen kapjuk:
$$\langle \kappa_1 \rangle = \frac{d\Phi(s)}{ds} \Big |_{\substack{ s=0 }} \hspace{0.5cm}, ..., \hspace{0.5cm} \langle \kappa_k \rangle = (-1)^k\frac{d^k\Phi}{ds^k}\Big |_{\substack{s = 0}}$$
Az első kumulánsokat könnyű kiszámolni, s egyszerű értelműk van
$$\langle \kappa_1\rangle = \langle n\rangle,  \hspace{0.5cm} \langle \kappa_2\rangle = \langle n^2\rangle -\langle n\rangle ^2,$$
Feladatok:\\
$(i)$ Határozzuk meg a kumulánsgeneráló függvényt az előadáson tárgyalt sorbanaállási problémára, s számítsuk ki az első két kumulánst! Vegyük észre, hogy az átlagos sorhossz és annak szórása lényegesen egyszerűbben kapható meg így, mint ha a momentumgeneráló fuüggvényen keresztül számoltuk volna.\\
$(ii)$Számítsuk ki a 3. kumulánst $(\kappa_3)$ a momentumokon keresztül! Mi lesz $\kappa_3$ értéke, ha n eloszlásfüggvénye szimmetrikus $ (P_n = P_{-n})$?
\subsection*{A feladat megoldása}


\newpage
\section*{3. feladat}
\subsection*{A feladat szövege}
Meredek hegyoldalban függőlegesen l távolságra vannak a kapaszkodoók. A hegymászó w rátával lép felfelé, s $w_0$ annak a rátaája, hogy visszacsűszik az $nl = 0$ szintre, s onnan folytatja a mászaást [ez a probléma példája az ún. újjrakezdési (reset) folyamatoknak].
\\Feladatok:\\
$(i)$Írjuk fel az egyenletet, amely meghatározza, hogy a hegymászó milyen $P_n$ valószínűséggel van nl magasságban!\\
$(ii)$ Határozzuk meg, hogy átlagosan milyen magasra jut a hegymászó!
\subsection*{A feladat megoldása}



\end{document}