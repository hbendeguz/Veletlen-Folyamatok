\documentclass[12pt]{article}
 \usepackage[margin=1in]{geometry} 
\usepackage{amsmath}
\usepackage[utf8]{inputenc}
%\usepackage[ utf8 ]{inputenc}
\def\doubleunderline#1{\underline{\underline{#1}}}
\newcommand{\angstrom}{\text{\normalfont\AA}}
\usepackage[magyar]{babel}
\newcommand{\N}{\mathbb{N}}
\newcommand{\Z}{\mathbb{Z}}
\usepackage{t1enc}
\usepackage{placeins}
\usepackage{caption}
\usepackage{textcomp}
\usepackage[utf8]{inputenc}
\usepackage[T1]{fontenc}
\title{Véletlen fizikai folyamatok, hatodik házi feladat}
\author{Horváth Bendegúz}

\begin{document}
\maketitle
\section*{1. feladat}
\subsection*{A feladat szövege}
Lokalizált mágneses momentum ($\mu$) z-tengely irányú B mágneses térben van. A mágneses
momentum ($\mu$) a tér irányában $\pm\mu B$ értéket vehet fel, s ezekben az állapotokban energiája $ \mp\mu BB$. A mágneses momentum T hőmérsékletű könyezettel van egyensúlyban, s kölcsönhatás eredményeképpen $\mu$ billeg a $-\mu B$ és $+\mu B$ állapotok között $ (-\mu B \leftrightarrow  +\mu B)$.\\
$(i)$ Írjuk fel a folyamat master egyenletét!\\
$(ii)$ Válasszunk olyan átmeneti rátákat, amelyek kielégítik a részletes egyensúly elvét!\\
$(iii)$ Határozzuk meg az átlagos mágneses momentum $\langle \mu (t)\rangle$ időfejlődését, ha kezdetben (t = 0) a mágneses momentum a z-tengely pozitív irányába mutatott.
\subsection*{A feladat mgoldása}




\newpage
\section*{2. feladat}
\subsection*{A feladat szövege}
Legyen egy egész értékeket felvevő stochasztikus vátozó, n, momentum-generátor fuüggvénye G(s). A normalizációból következik, hogy G(0) = 1, s n momentumai G deriváltján keresztül kifejezhetők:
$$\langle n\rangle = - \frac{dG(s)}{ds} \Big |_{\substack{ s=0 }}\hspace{0.7cm}, ...,\hspace{0.7cm} \langle n^k \rangle = (-1)^k\frac{d^kG(s)}{ds^k} \Big |_{\substack{ s=0 }}$$
A kumuláns generátor függvény a momentum-generáor függvény logaritmusa,
$$ \Phi (s) = \ln{G(s)}$$








\end{document}