\documentclass[12pt]{article}
 \usepackage[margin=1in]{geometry} 
\usepackage{amsmath}
\usepackage[utf8]{inputenc}
%\usepackage[ utf8 ]{inputenc}
\def\doubleunderline#1{\underline{\underline{#1}}}
\newcommand{\angstrom}{\text{\normalfont\AA}}
\usepackage[magyar]{babel}
\newcommand{\N}{\mathbb{N}}
\newcommand{\Z}{\mathbb{Z}}
\usepackage{t1enc}
\usepackage{placeins}
\usepackage{caption}
\usepackage{textcomp}
\usepackage[utf8]{inputenc}
\usepackage[T1]{fontenc}
\title{Véletlen fizikai folyamatok, hatodik házi feladat}
\author{Horváth Bendegúz}

\begin{document}
\maketitle
\section*{1. feladat}
\subsection*{A feladat szövege}
Lokalizált mágneses momentum ($\mu$) z-tengely irányú B mágneses térben van. A mágneses
momentum ($\mu$) a tér irányában $\pm\mu B$ értéket vehet fel, s ezekben az állapotokban energiája $ \mp\mu_BB$. A mágneses momentum T hőmérsékletű könyezettel van egyensúlyban, s kölcsönhatás eredményeképpen $\mu$ billeg a $-\mu B$ és $+\mu B$ állapotok között $ (-\mu B \leftrightarrow  +\mu B)$.\\
$(i)$ Írjuk fel a folyamat master egyenletét!\\
$(ii)$ Válasszunk olyan átmeneti rátákat, amelyek kielégítik a részletes egyensúly elvét!\\
$(iii)$ Határozzuk meg az átlagos mágneses momentum $\langle \mu (t)\rangle$ időfejlődését, ha kezdetben (t = 0) a mágneses momentum a z-tengely pozitív irányába mutatott.
\subsection*{A feladat megoldása}
$(i)$ A rendszer két állpot között változik, így ezekre az állapotokra felírva a master egyenlet:
$$\partial _tP_{-\mu}(t) = -w_0P_{-\mu}(t) + wP_{mu}(t)$$
$$\partial_t P_{\mu}(t)= w_0P_{-\mu}(t) - w P_{\mu}(t)$$
Az egyenletekben $P_{\pm\mu}(t)$ a $\pm\mu$ állapotban tartózkodás valószínűségét jelöli. \\
(ii) Először próbáljuk meg megoldani a sajátértékegyenletet:
\[
\partial_t
\begin{pmatrix}
P_{-\mu}(t)\\
p_{+\mu}(t)
\end{pmatrix}=
\begin{pmatrix}
-w_0& w\\
w_0 & -w
\end{pmatrix}
\begin{pmatrix}
P_{-\mu}(t)\\
p_{+\mu}(t)
\end{pmatrix}
\]

Az Ising-spinrendszerhez hasonlóan, a  $P^{(e)}_{\pm\mu} = \frac{1}{Z}e^{-\beta E(\mp \mu B)}$. A sajátértékek a számolás után a következőknek adódnak: $\lambda_1 = 0$, ez az egyensúlyi állapothoz tarozik, és $\lambda_2 = -(w_0 + w)$. Ezek után a részletes egyensúlyi elvét kielégítő átmeneti rátákat kell meghatározni. Legyen 
$$\frac{w_{n, n'}}{w_{n', n}} = e^{-\beta(E_{n'}-E_{n})}  $$
Ha $\Delta E = (E_{n'}-E_{n})> 0$, akkor az átmeneti ráta
$$w_{n',n} =\frac{1}{\tau}e^{-\beta(E_{n'}-E_{n})} .$$
A $\Delta E = \pm 2\mu_B B$ és pozitív, ha felfelé álló állapotból lefelé álló állapotba ugrott a mágneses momentum. Ha $\Delta E < 0$, akkor $w_{n',n} = \frac{1}{\tau}$, $\tau$ az urgások közti idő. Így 
$$w_0 = \frac{1}{\tau}$$
$$w= \frac{1}{\tau}e^{-\beta 2 \mu_B B}$$
megkaptuk az átmeneti rátákat.\\
(iii) Az előző feladatrészben kiszámoltuk a sajátértékeket, így az állapotban tartózkodás valószínűségét leíró vektort a következő módon kaphatjuk meg:
$$\vec{P}(t) = \vec{P}^{(e)} + e^{\lambda_2 t} a_2 \vec{P}_2$$
Az $a_2$ együtthatót a kezdeti feltételből határozhatjuk meg, ami esetünkben
$$\vec{P}(0) = (0, 1)^T = \vec{P}^{(e)} + a_2\vec{P}_2.$$
$\vec{P}_2 = (1, -1)^T$ sajátvektor\footnote{Megjegyzés: $T$-vel a transzponálást jelöltem.}, így behelyettesítve az egyenletbe
$$a_2 = -\frac{1}{Z}e^{-\beta \mu_B B}$$
Tehát
\[
 \vec{P}(t) = \frac{1}{Z}
\begin{pmatrix}
e^{-\beta \mu_B B}\\
e^{+\beta \mu_B B}
\end{pmatrix}
-\frac{1}{Z}e^{-\beta \mu_B B} e^{-\frac{1+\frac{1}{\tau}e^{-\beta 2 \mu_B B}}{\tau}t}
\begin{pmatrix}
1\\
-1
\end{pmatrix}
\]
Nekünk viszont a feladat teljes megoldásához a mágnesezettség várható értéke kell, amit a következő módon kaphtaunk meg:
$$\langle \mu(t)\rangle = (-1, 1)^T \vec{P}(t).$$
$$\langle \mu(t)\rangle = \frac{1}{Z}\Big [ -e^{-\beta \mu_B B}+2e^{-\beta \mu_B B}\cdot  e^{-\frac{1+\frac{1}{\tau}e^{-\beta 2 \mu_B B}}{\tau}t} +e^{\beta \mu_B B} \Big ]$$








\newpage


\section*{2. feladat}
\subsection*{A feladat szövege}
Legyen egy egész értékeket felvevő stochasztikus vátozó, n, momentum-generátor fuüggvénye G(s). A normalizációból következik, hogy G(0) = 1, s n momentumai G deriváltján keresztül kifejezhetők:
$$\langle n\rangle = - \frac{dG(s)}{ds} \Bigg |_{\substack{ s=0 }}\hspace{0.7cm}, ...,\hspace{0.7cm} \langle n^k \rangle = (-1)^k\frac{d^kG(s)}{ds^k} \Bigg |_{\substack{ s=0 }}$$
A kumuláns generátor függvény a momentum-generáor függvény logaritmusa,
$$ \Phi (s) = \ln{G(s)}$$
s a kumulánsokat a következőképpen kapjuk:
$$\langle \kappa_1 \rangle = \frac{d\Phi(s)}{ds} \Bigg |_{\substack{ s=0 }} \hspace{0.5cm}, ..., \hspace{0.5cm} \langle \kappa_k \rangle = (-1)^k\frac{d^k\Phi}{ds^k}\Bigg |_{\substack{s = 0}}$$
Az első kumulánsokat könnyű kiszámolni, s egyszerű értelműk van
$$\langle \kappa_1\rangle = \langle n\rangle,  \hspace{0.5cm} \langle \kappa_2\rangle = \langle n^2\rangle -\langle n\rangle ^2,$$
Feladatok:\\
$(i)$ Határozzuk meg a kumulánsgeneráló függvényt az előadáson tárgyalt sorbanaállási problémára, s számítsuk ki az első két kumulánst! Vegyük észre, hogy az átlagos sorhossz és annak szórása lényegesen egyszerűbben kapható meg így, mint ha a momentumgeneráló fuüggvényen keresztül számoltuk volna.\\
$(ii)$Számítsuk ki a 3. kumulánst $(\kappa_3)$ a momentumokon keresztül! Mi lesz $\kappa_3$ értéke, ha n eloszlásfüggvénye szimmetrikus $ (P_n = P_{-n})$?
\subsection*{A feladat megoldása}
A momentum-generátor függvényünk a következő alakú:
$$G(s) = \sum^{\infty}_{n = 0}e^{-sn}P_n$$
$P_n$-et tudjuk az órán felírt megoldásból,
%$$P_n = \frac{\lambda_n\lambda_{n-1}\text{...}\lambda_0}{\mu_n\mu_{n-1}\text{...}
%\mu_1}P_0$$
%$$P_{-n} =\frac{\mu_0\mu_{-1}{...}\mu_{-(n+1)}}{\lambda_{-1}\lambda_{-2}{...}\lambda_{-n}}P_0$$
Esetünkben a momentum-generátor függvénynél csak 0-tól megy a szummázás, így onnantól fog kelleni a $P_n$. Számoljuk ki az első kumulánst:
$$\langle \kappa_1 \rangle = -\frac{d\Phi(s)}{ds}\Bigg |_{s = 0} =- \frac{d \ln(G(s))}{ds}\Bigg |_{s = 0} =-\Bigg (\frac{1}{G(s)}\frac{dG(s)}{ds}\Bigg )\Bigg |_{s = 0} = -\frac{dG(s)}{ds} \Bigg |_{s = 0} =  \sum^\infty_{n=0} nP_n$$
%$$=nP_0 + (1-P_0)=P_0(n-1)+ 1 $$ 
A második kumuláns:
$$\langle \kappa_2 \rangle = \frac{d}{ds}\frac{d}{ds}\ln(G(s))\Bigg |_{s = 0}= \Bigg (\frac{d}{ds}\frac{1}{G(s)}\frac{dG(s)}{ds}+\frac{1}{G(s)}\frac{d^2G(s)}{ds^2}\Bigg )\Bigg |_{s = 0} = \Bigg(- \frac{1}{G^2(s)}\frac{dG(s)}{ds}+  \frac{1}{G(s)}\frac{d^2G(s)}{ds^2}  \Bigg )\Bigg |_{s = 0} = $$
$$=- \langle n \rangle ^2+\sum^\infty_{n=0} n^2P_n = -\langle n \rangle^2 +\langle n^2 \rangle $$\\
\par
 A sorbanállási problémánal, ahol $\lambda_n = \lambda$ és $\mu_n = \mu$ volt, a momentum-generátor függvény egyensúlyi alakjára a következőt kaptuk:
 $$G_e(s)= \frac{(1-e^s)P_0}{1-e^s+q_qe^{-s}} = \frac{P_0}{1-qe^{-s}} $$
 ahol $q = \frac{\lambda}{mu}$. Ennek véve a logaritmusát:
 $$\Phi(s) = \ln G(s) = \ln{P_0}-\ln{(1-qe^{-s})}$$
Ezt deriváljuk $s$ szerint, majd kiértékeljuük az $s = 0$ helyen:
$$\langle \kappa_1\rangle = \frac{d\Phi(s)}{ds}\Bigg |_{s = 0} = \frac{1}{1-qe^{-s}}qe^{-s} \Bigg |_{s = 0} = \frac{q}{1-q} = \langle n\rangle$$
Tényleg azt kaptuk, mint a momentum-generátor függvény deriválásával. Megnézem a második deriváltat:
$$\langle \kappa_2 \rangle = -\frac{d^2\Phi(s)}{ds^2}\Bigg |_{s = 0} = -\frac{d}{ds}\frac{qe^{-s}}{1-qe^{-s}}\Bigg |_{s = 0} =-\Bigg (\frac{-qe^{-s}}{1-qe^{-s}} +\frac{q^2e^{-2s}}{(1-qe^{-s})^2}\Bigg )\Bigg |_{s = 0 } = $$
$$ = \frac{q}{1-q} - \frac{q^2}{(1-q)^2} = \langle n^2\rangle-\langle n\rangle ^2$$
És megint ugyanazt kaptuk.\\
$(ii)$ Számítsuk ki $\langle \kappa_3\rangle$-at:
$$\langle \kappa_3 \rangle =  \frac{d}{ds} \Bigg (\frac{-qe^{-s}}{1-qe^{-s}} +\frac{q^2e^{-2s}}{(1-qe^{-s})^2}\Bigg )\Bigg |_{s = 0 }= \Bigg (\frac{+qe^{-s}}{1-qe^{-s}}-\frac{q^2e^{-2s}}{(1-qe^{-s})^2}+$$
$$+\frac{2q^2e^{-2s}}{(1-qe^{-s})^3} + \frac{qe^{-s}}{(1-qe^{-s})^2}\Bigg )\Bigg |_{s= 0} =\frac{q}{1-q}\Bigg ( 1 - \frac{q}{1-q}-\frac{q^2}{1-q} +$$
$$+\frac{2q^2}{(1-q)^2}+ \frac{1}{(1-q)^2}\Bigg ) = $$
Ha az eloszlásfüggvény szimmetrikus, $\kappa_3 = 0$, mert akkor a várható értéknek is 0-nak kell lennie, és azzal kiemeltünk a $\kappa_3$ számításánál.













\newpage
\section*{3. feladat}
\subsection*{A feladat szövege}
Meredek hegyoldalban függőlegesen l távolságra vannak a kapaszkodoók. A hegymászó w rátával lép felfelé, s $w_0$ annak a rátaája, hogy visszacsűszik az $nl = 0$ szintre, s onnan folytatja a mászást [ez a probléma példája az ún. újrakezdési (reset) folyamatoknak].
\\Feladatok:\\
$(i)$Írjuk fel az egyenletet, amely meghatározza, hogy a hegymászó milyen $P_n$ valószínűséggel van nl magasságban!\\
$(ii)$ Határozzuk meg, hogy átlagosan milyen magasra jut a hegymászó!
\subsection*{A feladat megoldása}
A master-egyenlet:
$$ \partial_tP_n(t) = -(w+w_0)P_n(t) + wP_{n-1}(t) $$
$$\partial_tP_0(t ) = -(w)P_0(t) + \sum^\infty_{n = 1}w_0P_n(t)$$
Így $P_0(t)$-ra kapunk egy differenciálegynelet:
$$\partial_tP_0(t) = -(w+w_0)P_0(t) + w_0$$
Megoldása:
$$P_0(t) = A\cdot e^{-(w+w_0)t} + w_0\cdot t$$
$$\partial_t P_1(t) =-(w+w_0)P_{1}(t) +w\cdot \Big[A\cdot e^{-(w+w_0)t} + w_0\cdot t\Big ] $$
De itt már nagyon elbonyolódik a differenciálegyenlet, így más megoldási módot kell keresni. \\
Használjuk a generátor-függvény formalizmust:
$$G(s,t) = \sum^\infty_{n=0}e^{-sn}P_{n}(t)$$
Legyen 
$$q = \frac{w}{w_0}\text{ és }\tau = \frac{1}{w_0}$$
Így felírva a master egyenletet:
$$\tau\partial_t G(s, t) = \sum^\infty_{n=0}e^{-sn}P_n= -qP_0+\sum^\infty_{n=1}e^{-s}P_n+\sum^\infty_{n = 1}-(1+q)e^{-sn}P_n+ qe^{-sn}P_{n-1}=$$
$$=-qP_0 +\Big [-(1+q)+ qe^{-s}+ e^{-s}\Big ]G(s, t)$$





Stacionárius megoldás esetén a baloldala nulla, így felírható $G_e(s, t)$:
$$G_e(s) = \frac{qP_0}{-(1+q)+ qe^{-s}+ e^{-s}}$$
Az előző feladatban láthattuk, hogy logaritmusát véve $G_(s)$-nek és utána $s$ szerint deriváljuk, kiértékeljük az $s =0$ helyen,  megkaphatjuk a várható értéket.
tegyük meg:
$$\ln{G_e(s)} = \ln{(qP_0)}- \ln{(-(1+q)+ qe^{-s}+ e^{-s})}$$
$$\frac{d}{ds}\ln{G_e(s)}= -\frac{1}{(-(1+q)+ qe^{-s}+ e^{-s})}(-q-1)= \frac{q+1}{(-(1+q)+ qe^{-s}+ e^{-s})}$$
Ezt kiértékeljük a 0 helyen:
$$\frac{d}{ds}\ln{G_e(s)}\Big |_{s=0} = \frac{-P_0}{qP_0-1}+\frac{q}{q-(1+q)} $$
$P_0$ értékét megkaphatjuk a $G(s=0) =1$ feltételből:

\end{document}