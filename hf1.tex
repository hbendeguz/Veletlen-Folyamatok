\documentclass[12pt]{article}
 \usepackage[margin=1in]{geometry} 
\usepackage{amsmath}
\usepackage[utf8]{inputenc}
%\usepackage[ utf8 ]{inputenc}
\usepackage[magyar]{babel}
\newcommand{\N}{\mathbb{N}}
\newcommand{\Z}{\mathbb{Z}}
\usepackage{t1enc}
\usepackage{placeins}
\usepackage{caption}
\usepackage[utf8]{inputenc}
\usepackage[T1]{fontenc}
\title{Véletlen fizikai folyamatok, első házi feladat}
\author{Horváth Bendegúz}

\begin{document}
 
%\renewcommand{\qedsymbol}{\filledbox}
%Good resources for looking up how to do stuff:
%Binary operators: http://www.access2science.com/latex/Binary.html
%General help: http://en.wikibooks.org/wiki/LaTeX/Mathematics
%Or just google stuff
 
\maketitle

\section*{2. feladat}
\subsection*{A feladat szövege}
Feldobott érme leesése után egyenlő valószínűséggel fej (F) vagy írás (I).
\\
(a) Dobjuk fel az érmét kétszer. Milyen valószínűséggel kapunk két fejet (FF), illetve fej-írás (FI) sorrendet? Ugyanaz a két valószínűség?
\\ 
(b) Játsszuk a következő játékot! Addig dobálunk, amíg vagy két fej (FF - én nyerek), vagy fej-írás (FI - te nyersz) jön ki. Igazságos ez a játék?
\subsection*{A feladat megoldása}
(a) Két kimenetelű rendszernek tekinthető a feldobott érme rendszere, egyforma valószínűségű végkimenetelekkel, így $P(\text{F}) = \frac{1}{2} $ a fej valószínűsége, és  $P(\text{I}) = \frac{1}{2} $ az írás valószínűsége. Két dobás kimenetele a valószínűségek szorzata: $$P(\text{F,F}) = P(\text{F})\cdot P(\text{F}) = \frac{1}{2}\cdot \frac{1}{2} = \frac{1}{4} $$
$$P(\text{F,I}) = P(\text{F})\cdot P(\text{I}) = \frac{1}{2}\cdot \frac{1}{2} = \frac{1}{4} $$
$$P(\text{I,I}) = P(\text{I})\cdot P(\text{I}) = \frac{1}{2}\cdot \frac{1}{2} = \frac{1}{4} $$
$$P(\text{I,F}) = P(\text{I})\cdot P(\text{F}) = \frac{1}{2}\cdot \frac{1}{2} = \frac{1}{4} $$
\\
A valószínűség $P = \frac{\text{kedvező elemi események száma}}{\text{összes elemi események száma}}$ értelmezésével az elemi események ${FF, IF, FI, II}$, kijön az előbbi végeredmény, hogy a két valószínűség egyforma, értékük $\frac{1}{4}$.\\
(b)
\end{document}