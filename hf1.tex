\documentclass[12pt]{article}
 \usepackage[margin=1in]{geometry} 
\usepackage{amsmath}
\usepackage[utf8]{inputenc}
%\usepackage[ utf8 ]{inputenc}
\usepackage[magyar]{babel}
\newcommand{\N}{\mathbb{N}}
\newcommand{\Z}{\mathbb{Z}}
\usepackage{t1enc}
\usepackage{placeins}
\usepackage{caption}
\usepackage[utf8]{inputenc}
\usepackage[T1]{fontenc}
\title{Véletlen fizikai folyamatok, első házi feladat}
\author{Horváth Bendegúz}

\begin{document}
 
%\renewcommand{\qedsymbol}{\filledbox}
%Good resources for looking up how to do stuff:
%Binary operators: http://www.access2science.com/latex/Binary.html
%General help: http://en.wikibooks.org/wiki/LaTeX/Mathematics
%Or just google stuff
 
\maketitle

\section*{2. feladat}
\subsection*{A feladat szövege}
Feldobott érme leesése után egyenlő valószínűséggel fej (F) vagy írás (I).
\\
(a) Dobjuk fel az érmét kétszer. Milyen valószínűséggel kapunk két fejet (FF), illetve fej-írás (FI) sorrendet? Ugyanaz a két valószínűség?
\\ 
(b) Játsszuk a következő játékot! Addig dobálunk, amíg vagy két fej (FF - én nyerek), vagy fej-írás (FI - te nyersz) jön ki. Igazságos ez a játék?
\subsection*{A feladat megoldása}
(a) Két kimenetelű rendszernek tekinthető a feldobott érme rendszere, egyforma valószínűségű végkimenetelekkel, így $P(\text{F}) = \frac{1}{2} $ a fej valószínűsége, és  $P(\text{I}) = \frac{1}{2} $ az írás valószínűsége. Két dobás kimenetele a valószínűségek szorzata: $$P(\text{F,F}) = P(\text{F})\cdot P(\text{F}) = \frac{1}{2}\cdot \frac{1}{2} = \frac{1}{4} $$
$$P(\text{F,I}) = P(\text{F})\cdot P(\text{I}) = \frac{1}{2}\cdot \frac{1}{2} = \frac{1}{4} $$
$$P(\text{I,I}) = P(\text{I})\cdot P(\text{I}) = \frac{1}{2}\cdot \frac{1}{2} = \frac{1}{4} $$
$$P(\text{I,F}) = P(\text{I})\cdot P(\text{F}) = \frac{1}{2}\cdot \frac{1}{2} = \frac{1}{4} $$
\\
A valószínűség $P = \frac{\text{kedvező elemi események száma}}{\text{összes elemi események száma}}$ értelmezésével az elemi események ${FF, IF, FI, II}$, kijön az előbbi végeredmény, hogy a két valószínűség egyforma, értékük $\frac{1}{4}$.\\
(b) A dobások kimenetelei nem függnek az előző dobás végeredményétől. Az játékban az első fej kimenetel után a következő dobással valaki nyer, vagy FI vagy FF sorozat lesz belőle. Mind a kettőnek 0.5 a valószínűsége,mind a kettőnknek ugyanannyi esélye van nyerni, így igazságos a játék.
\newpage
\section*{3. feladat}
\subsection*{A feladat szövege}
Egydimenziós mozgást végző részecske $\tau$ időközönként véletlen irányú erő hatására előző helyzetétől $l$ távolságra ugrik (egyenlő $p_{+} = p_{-} =\frac{1}{2} $ valószínűséggel jobbra vagy balra). A részecske az $x_0 = 0$ pontból indul.
Határozzuk meg a $ t = N \tau$ idő alatti elmozdulás és az elmozdulás négyzetének átlagát, $\langle x_t\rangle$-t és $\langle x^2_t \rangle$-t!
Vizsgáljuk a fenti problémát $p_+ = 3p_-$ esetre és számítsuk ki az $\langle x_t \rangle $, $\langle x^2_t\rangle$, és $\langle x^2_t\rangle-\langle x_t \rangle ^2$ átlagokat!
\subsection*{A feladat megoldása}
A lépések összessége egy $N$ és $p$, paraméterű binominális eloszlást követ,ahol $N$ az események száma, $p$ pedig a valószínűsége az egyik eseménynek. Így kiszámolhatjuk, hogy mennyi a jobbra lépések számának várható értéke, mennyi a balra lépések számának várható értéke, és ebből számolható az $x_t$ távolság várható értéke. A binominális eloszlás képlete általános esetre:
$$P(X = k) = {{N}\choose{k}}p^k(1-p)^{N-k}.$$
Várható értéke:
$$ \langle X \rangle = Np .$$
A mi feladatunkban $N$ db ugrás van, egy ugrással $\pm l$ távolságot tesz meg. A pozitív irányba történő ugrások számának várható értéke $Np_+ = \frac{N}{2}$, amivel $\frac{Nl}{2}$ távolságot tesz meg jobbra. A negatív irányba való ugrásokkal számolva is ez jön ki, csak más előjellel. A kettő összege $0$-t ad. Így:
$$\langle x_t \rangle = 0. $$
Az $\langle x^2_t \rangle $-et a szórásból számolhatjuk,
$$ \sigma = \sqrt{\langle x^2_t \rangle-\langle x_t \rangle ^2},$$ binominális eloszlásnál $\sigma = \sqrt{np(1-p)}$. A képletbe behelyettesítve $\sigma_{\text{lépések}} = \sqrt{N}/2$ jön ki, ezt átírva távolságra $\sigma_{x_t} = \sqrt{N}\cdot l /2$, aminek a négyzete megegyezik $\langle x^2_t \rangle $-vel.
$$\langle x^2_t \rangle = \frac{N\cdot l^2}{4}.$$
A másik esetben, amikor $p_+=3p_-=0.75$ a számolási módszerek hasonlóak, az eredmények az új paraméterekkel:
$$\langle x_t \rangle = l\cdot N\cdot \Big(\frac{3}{4}-\frac{1}{4}\Big ) = \frac{N\cdot l}{2},$$
$$\sigma^2 = {\langle x^2_t \rangle-\langle x_t \rangle ^2} = l^2\cdot N\cdot \frac{3}{16},$$
$$\langle x^2_t \rangle = \sigma^2 + \langle x_t \rangle ^2  = l^2\Big ( \frac{N^2}{4} + \frac{3N}{16} \Big ) = l^2\frac{4N^2 + 3N}{16}$$
\newpage
\section*{4. feladat}
\subsection*{A feladat szövege}
Vizsgáljuk a Brown mozgás előadáson tárgyalt, Einstein féle leírását, s legyen sodródás is a rendszerben (szél fúj a víz felett). Ekkor a $\tau$ időnként megtett ugrások hosszának ($\Delta$) valószínűséegi eloszlása nem szimmetrikus $\Phi(-\Delta) \neq \Phi(\Delta)$, s várhatóan $\Delta =   \Delta\Phi(\Delta)d\Delta\neq 0$.\\
(a) Írjuk fel a Chapman-Kolmogorov egyenletet, s deriváljuk a részecske megtalálási valószínűségét, $P (x, t)$-t meghatározó Fokker-Planck egyenletet! Mennyiben különbözik ez az egyenlet az előadáson tárgyalt diffúziós egyenlettől?
\\
(b) Írjuk fel az egyenlet megoldását arra az esetre, ha a virágporszem az origóból indul!
\subsubsection*{A feladat megoldása}
(a) A Chapman-Kolmogorov egyenlet a következőképpen néz ki:
$$P(x,t+\tau) = \int_{\infty}^{\infty}\Phi(\Delta)P(x-\Delta, t)d\Delta.$$
Ez azon valószínűségek összege, hogy a részecske $x-\Delta$-ban volt $t$ időpilanatban és pont $\Delta$-t ugrott. Sorba fejtve:
$$P(x,t)+\frac{\partial P(x,t)}{\partial t}\tau =\int_{\infty}^{\infty}\Phi(\Delta)d\Delta P(x,t) -\int_{\infty}^{\infty}\Phi(\Delta)\Delta d\Delta\frac{\partial P(x,t)}{\partial x} + \frac{1}{2}\int_{\infty}^{\infty}\Phi(\Delta)\Delta^2d\Delta\frac{\partial^2P(x,t)}{\partial x^2}  .$$
A valószínűségeloszlás szimmetriája sérül, de az $\int_{\infty}^{\infty}\Phi(\Delta)d\Delta$ integrál továbbra is egyet ad, így az egyenlet rövidebb lesz:
$$\frac{\partial P(x,t)}{\partial t}\tau = -\int_{\infty}^{\infty}\Phi(\Delta)\Delta d\Delta\frac{\partial P(x,t)}{\partial x} + \frac{1}{2}\int_{\infty}^{\infty}\Phi(\Delta)\Delta^2d\Delta\frac{\partial^2P(x,t)}{\partial x^2}$$
$$\frac{\partial P(x,t)}{\partial t}\tau = -\overline \Delta\cdot\frac{\partial P(x,t)}{\partial x} + \frac{\overline{\Delta ^2}}{2}\cdot\frac{\partial^2P(x,t)}{\partial x^2}$$
Az így kapott egyenlet annyiban különbözik az előadáson tárgyalt diffúziós egyenlettől, hogy benne maradt a $\Delta$ várható értékével szorzódó, helyszerint egyszeresen derivált valószínűség, negatív előjellel.
\end{document}