\documentclass[12pt]{article}
 \usepackage[margin=1in]{geometry} 
\usepackage{amsmath}
\usepackage[utf8]{inputenc}
%\usepackage[ utf8 ]{inputenc}

\newcommand{\angstrom}{\text{\normalfont\AA}}
\usepackage[magyar]{babel}
\newcommand{\N}{\mathbb{N}}
\newcommand{\Z}{\mathbb{Z}}
\usepackage{t1enc}
\usepackage{placeins}
\usepackage{caption}
\usepackage{textcomp}
\usepackage[utf8]{inputenc}
\usepackage[T1]{fontenc}
\title{Véletlen fizikai folyamatok, második házi feladat}
\author{Horváth Bendegúz}


\begin{document}
\maketitle



\section*{1. feladat}

\subsection*{ A feladat szövege} A következőfeladatban leírt Perrin kísérlet megértéséhez oldjuk meg előszőr a két-dimenziós Brown mozgás
egy egyszerűváaltozatát: $l$ rácsállandójúnégyzetrácson egy részecske $\tau$ időközönként, egyenlő valószínűséggel ugrik a négy szomszédos rácspont egyikébe. A részecske $ (x0 = 0, y0 = 0)$ pontból indul.
Határozzuk meg a $t = N\tau $ idő alatti várható elmozdulást,  $\sqrt{\langle r^2 \rangle }= \sqrt{\langle x^2_t \rangle+ \langle y^2_t\rangle}$-t!

 
\subsection*{A feladat megoldása} Az órán megbeszltekhez hasonlóan a valószínűségi változó várható értékét felírhatjuk a valószínűségek és értékük összegeként, csak itt most két dimenziós vektorok összege lesz:
$$\langle e_i \rangle = \frac{1}{4}\big ( (0, 1)+(0,-1)+(1,0)+(-1,0)\big ) =0, $$
$$ x_N = l\cdot \sum_{i=1}^N e_i,$$
ahol $x_N$ a részecske helye. A távolság a helyvektor önmagával vett skalárisszorzata:
$$\langle x^2_N\rangle = l^2\sum_{j=1}^Ne_j\sum_{i=1}^N \langle e_i \rangle = l^2\sum_{i,j=1}^N \langle e_i e_j\rangle = l^2 \Big (\sum_{i=1}^N \langle e^2_i \rangle +\sum_{i,j \neq 1}^N\langle e_i e_j \rangle \Big ) = l^2 \sum_{i=1}^N \langle e^2_i \rangle = {l^2N}$$
Így a távolság várható értéke:
$$\sqrt{\langle r^2 \rangle} = l\sqrt{N} $$
 
\newpage
\section*{2. feladat}

\subsection*{ A feladat szövege}
Perrin kísérletében kolloid részecskék mozgását vizsgáalták híg, vizes oldatban. A részecskék sugara
$a = 0.52\mu m, \tau = 30s$-ként mérték a helyzetüket, s az ábrán látható négyzetrács rácsállandóa $3.125\mu m$. Becsüljük meg a kolloidrészecskék diffúziós együtthatóját kétféleképpen: (a) a kezdő és a végpont közötti elmozdulásból, feltételezve, hogy a mozgás diffúziv, és (b) a $\tau$ idő alatti ugráshosszok négyzetének átlagából!

 
 \subsection*{A feladat megoldása}
$(a) $ A feladat megoldásához meg kellett számolni mind a 3 részecskének a kiindulási helyüktől megtett távolságot és a lépések számát. Az így kapott eredményeket a következő táblázatban foglalom össze:
\begin{center}
\begin{tabular}{|c|c|c|c|}\hline

részecske & lépések száma & távolság négyzete [$ m$]& t [s]\\ \hline 
baloldali &46 &$2.225 \cdot 10^{-9}$ &1380 \\ \hline
középső &30 & $2.769 \cdot 10^{-9}$& 900 \\ \hline
jobboldali &40  &$0.711\cdot 10^{-9}$ &1200   \\ \hline

\end{tabular}
\end{center}

Kihasználva a következő összefüggést a diffúziós együtthatóra:
$$R^2(t) = 2 D t.$$
$R(t)$ a kiindulásiponttól megtett távolsága,  a képletbe behelyettesítve a diffúzós állandók sorrendben $8.15\cdot 10^{-13} \frac{m^2}{s}$, $1.53\cdot 10^{-12} \frac{m^2}{s}$, és $2.96\cdot 10^{-13} \frac{m^2}{s}$ lettek. Ezeknek az átlaga $D = 8.83\cdot 10^{-13} \frac{m^2}{s}$. \\ \\
$(b)$ Ennél a feladatnál le kellett számolni egy bolyongásban az összes lépésnek a négyzetét, majd ennek az átlagát venni. Ezt megtettem, a számolásokat egy táblázatkezelő programban végeztem \cite{wik} . A  $D$ diffúziós együtthatót a következő módon lehet megkapni az ugráshosszok négyzetének átlagából:
$$D = \frac{\langle \Delta ^2 \rangle }{2\tau}.$$
$\Delta$ jelöli az ugráshosszakat, $\tau = 30 s$ esetünkben. Így, balról jobbra az egyes bolyongásokhoz tartozó diffuziós együtthatók: $1.36\cdot 10^{-12}\frac{m^2}{s}$, $6.24\cdot 10^{-13} \frac{m^2}{s}$ és $7.52\cdot 10^{-13} \frac{m^2}{s}$. \\ A diffúziós együtthatók átlaga: $D = 9.12\cdot 10^{-13} \frac{m^2}{s}$
\\\\
Bár az $(a)$ és  $(b)$ feladatoban egyes bolyongásokhoz tartozó $D$-k különböznek egymástól, az átlagok viszont egész közel vannak egymáshoz. Ezért a  $3$-as feladatban a kettőnek az átlagát fogom használni a becslésben.
\newpage

\section*{3. feladat} 
\subsection*{ A feladat szövege}
Használjuk a $(2)$ feladat eredményét, valamint a Brown mozgás Langevin féle leírásának eredményeképp
kapott kifejezést a kolloidrészecskék diffúziós együttthatójára, s becsüjük meg az Avogadro számot! A kolloidrészecskék sűrűségét tekinthetjük vízhez közelinek, a hőmérsékletet pedig szobahőmérsékletnek.

 
 \subsection*{A feladat megoldása}
 A megoldáshoz felhasználjuk a következő kifejezést, ami a Brown-mozgás Langevin féle levezetéséből kaptunk:
$$D = \frac{k_B T}{6\pi\eta a}$$
A kifejezésben $\eta$ a viszkozitás, $a$ a kolloidrészecske sugara, $T$ a hőmérséklet. Esetünkben $\eta$-t víz közeli sűrűségűnek vettem, $\eta = 8\cdot 10^{-4} Pa\cdot s$, $a = 5.2\cdot 10^{-7} m$, $T = 290 \text{\textdegree{K}}$. és $D$ az előző feladatból a kétféleképpen megkapott eredmény átlaga lett, $D = 8.97\cdot 10^{-13}\frac{m^2}{s}$. Ismert még, hogy $$k_B= \frac{R}{N_A},$$ ahol a $R$ az egyetemes gázállandó, $N_A$ pedig az Avogadro-szám. Így a megoldandó egyenlet:
$$ D = \frac{R\cdot T}{6\pi \eta a N_A} .$$
Megfelelő alakra rendezve és behelyettesítve az adatokat\cite{wik}, az Avogadro-számra $N_A = 3.461\cdot 10^{23} mol^{-1}$ jön ki, ami kicsivel több mint a fele az elméleti értéknek. 



\newpage
\section*{4. feladat}
\subsection*{ A feladat szövege}
Tegyük fel, hogy a kolloidrészecskék diffúziós együttthatójárajára kapott kifejezés extrapolálható molekuláris
szintre. Milyen értéket kapunk egy nem túlságosan nagy molekulekula vízben történő termális mozgásának diffúziós együtthatójára? Keressünk nagy (biológiai) molekulákat (DNS?), amelyekre a diffúziós együtható ismert, s hasonlítsuk össze értéküket a becsüt eredménnyel!

 
 \subsection*{A feladat megoldása}
  A diffúziós együtthatókat egy honlapon találtam\cite{coeff}, \cite{coeff2}.
Az órán levezetett képlet, amibe behelyettesíttem:
$$  D = \frac{k_B T}{6\pi\eta a}.$$

\begin{center}
\begin{tabular}{|c|c|c|c|c|}
\hline
Tapasztalati képlet & $a$ sugár & elméleti diffúziós együttható &számolt diffúziós együttható & T [\textdegree{K}]\\ \hline
CO2 &1.16 $ \angstrom $ & $1.92\cdot 10^{-9} m^2/s$ & $1.88\cdot 10^{-9} m^2/s$ & 295  \\ \hline
NH3 & 1.41 $ \angstrom $&$1.64\cdot 10^{-9} m^2/s$ & $1.47\cdot 10^{-9} m^2/s$   &285\\ \hline
H2 & 0.5 $ \angstrom $ &$4.5\cdot 10^{-9} m^2/s $& $4.36\cdot 10^{-9} m^2/s $&298 \\ \hline
$C_2H_6$ (etán) & 1.55 $ \angstrom $ & $1.2\cdot 10^{-9} m^2/s $ &$1.4\cdot 10^{-9} m^2/s $& 298 \\ \hline



\end{tabular}
\end{center}

A képletből számolt értékek megközelítik az elméleti értéket, habár a sugár kiszámolása nem mindig volt pontos. A nagyobb biologiai molekulákhoz sajnos nem találtam adatokat.


\begin{thebibliography}{9}
\bibitem{wik}
\texttt{http://hbendeguz.web.elte.hu/java/vélf2\_{}1}

\bibitem{coeff}
\texttt{http://www.thermopedia.com/content/696/}

\bibitem{coeff2}
\texttt{http://biofilmbook.hypertextbookshop.com/public\_{}version/artifacts/tables/ \\Module\_{}004/Table4-1\_{}DiffCoeffH2O.htm}

\end{thebibliography}






















\end{document}