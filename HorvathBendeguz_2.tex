\documentclass[12pt]{article}
 \usepackage[margin=1in]{geometry} 
\usepackage{amsmath}
\usepackage[utf8]{inputenc}
%\usepackage[ utf8 ]{inputenc}
\usepackage[magyar]{babel}
\newcommand{\N}{\mathbb{N}}
\newcommand{\Z}{\mathbb{Z}}
\usepackage{t1enc}
\usepackage{placeins}
\usepackage{caption}
\usepackage[utf8]{inputenc}
\usepackage[T1]{fontenc}
\title{Véletlen fizikai folyamatok, második házi feladat}
\author{Horváth Bendegúz}


\begin{document}
\maketitle
\section*{1. feladat}
\subsection*{ A feladat szövege} A következőfeladatban leírt Perrin kísérlet megértéséhez oldjuk meg előszőr a két-dimenziós Brown mozgás
egy egyszerűváaltozatát: $l$ rácsállandójúnégyzetrácson egy részecske $\tau$ időközönként, egyenlő valószínűséggel ugrik a négy szomszédos rácspont egyikébe. A részecske $ (x0 = 0, y0 = 0)$ pontból indul.
Határozzuk meg a $t = N\tau $ idő alatti várható elmozdulást,  $\sqrt{\langle r^2 \rangle }= \sqrt{\langle x^2_t \rangle+ \langle y^2_t\rangle}$-t!

 
\subsection*{A feladat megoldása} Bla
 
\newpage
\section*{2. feladat}
\subsection*{ A feladat szövege}
Perrin kísérletében kolloid részecskék mozgását vizsgáalták híg, vizes oldatban. A részecskék sugara
$a = 0.52\mu m, \tau = 30s$-ként mérték a helyzetüket, s az ábrán látható négyzetrács rácsállandóa $3.125mu m$. Becsüljük meg a kolloid részecskék diffúziós együtthatóját kétféleképpen: (a) a kezdő és a végpont közötti elmozdulásból, feltételezve, hogy a mozgás diffúziv, és (b) a $\tau$ idő alatti ugráshosszok négyzetének átlagából!

 
 \subsection*{A feladat megoldása}


\newpage

\section*{3. feladat} 
\subsection*{ A feladat szövege}
 
 \subsection*{A feladat megoldása}

\newpage
\section*{4. feladat}
\subsection*{ A feladat szövege}
 
 \subsection*{A feladat megoldása}

\end{document}