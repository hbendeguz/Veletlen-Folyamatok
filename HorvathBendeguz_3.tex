\documentclass[12pt]{article}
 \usepackage[margin=1in]{geometry} 
\usepackage{amsmath}
\usepackage[utf8]{inputenc}
%\usepackage[ utf8 ]{inputenc}

\newcommand{\angstrom}{\text{\normalfont\AA}}
\usepackage[magyar]{babel}
\newcommand{\N}{\mathbb{N}}
\newcommand{\Z}{\mathbb{Z}}
\usepackage{t1enc}
\usepackage{placeins}
\usepackage{caption}
\usepackage{textcomp}
\usepackage[utf8]{inputenc}
\usepackage[T1]{fontenc}
\title{Véletlen fizikai folyamatok, harmdik házi feladat}
\author{Horváth Bendegúz}


\begin{document}
\maketitle
\section*{1. feladat}
\subsection*{A feladat szövege}
Egy szobában T=20$\text{ \textdegree{C}}$ hőmérsékletű ideális gáznak tekinthető levegő van. Számítsuk
ki mennyi idő alatt jut el egy vízmolekula a szoba egyik végéből a másikba tisztán diffúziv mozgással.\\
Segítség:
A kinetikus elmélet a gázok diffúziós együtthatójára a következő kifejezést adja (és az eredményt érthetjük is a Brown mozgásról tanultak alapján):
$$ D = \frac{1}{3}l\overline{v}\hspace{1 cm}\Big (= \frac{l^2}{3l/\overline{v}}\approx\frac{(\Delta x)^2}{2\tau}\Big )$$
     ahol l a molekulák szabad úthossza, v pedig átlagos sebességük. A szabad úthosszt megbecsülhetjük a $l = 1/(n\pi d^2)$ kifejezésből, ahol n a molekulák koncentrációja és d a molekulák átmérője. Az átlagos sebességet pedig az ekvipartíció tételéből számolhatjuk.
A valóságban a szagok sokkal gyorsabban terjednek egy szobában, annak ellenére, hogy a megfelelő molekulák lényegesen nagyobbak és súlyosabbak a vízmolekulánál. Értjük ezt?


\subsection*{A feladat megoldása}
Felírva az ekvipartíció tételét megkaphatjuk a $v$ sebességet:
$$E = \frac{1}{2}m\big (v^2_x+v^2_y+v^2_z \big ) = \frac{3}{2}k_B T.$$
A kéletbe behelyettesítve $T = 293$, a levegő moláris tömege $18.015 \text{ }g/mol$, amiből egy mlekula tömege $\approx 18.015/8.31/10^3$ kg.
 $$\overline{v} = \sqrt{293\cdot 10^3 \frac{3}{2.16 }} \approx 638 \text{  }\frac{m}{s} $$
A vízmolekula szabad úthossza kell még, hogy a diffúziós együtthatóját megkaphassuk. A víz molekula átmérőjét $2.75\cdot10^{-10}\text{ }m$-nek vettem\cite{diam}. A levegő koncentrációját pedig $55 mol/l$-nek \cite{conc}.
Így a szabad úthossza:
$$l = \frac{1}{55 \pi\cdot 2.75^2\cdot 10^{-20}\cdot 6\cdot 10^{23}\cdot 10^3} = \approx 1.27 \cdot 10^{-10} \text{ m.}$$
A diffúziós együttható:
$$D = \frac{1}{3}\cdot 1.27\cdot 10^{-10}\cdot 638 = 2.7\cdot 10^{-8} \text{  }\frac{m^2}{s} .$$
Felhasználva az előző órákról a diffúziós üsszefüggést:
$$R(t) = 2Dt,$$
ahol a képletben $R(t)$ a távolság, $t$ az idő és $D$ a diffúziós együttható. A szobát egy $7 \times 5\times 3 \text{  }m^3$ téglatestnek tekintem, így az egyik sarkából $\sqrt{49+25+9} = 9.1 m$ utat kell megtennie, hogy átérjen a legtávolabb lévő sarkába. Behelyettesítve a fenti képlet átalakított verziójába :
$$t =  \frac{9.11\cdot 10^{10}}{2\cdot 1.27} = 3.58\cdot 10^{10}\text{ s}$$
A kapott végeredmény nagyságrendekkel nagyobb a tapasztaltaknál, tisztán diffúzióval nagyon sok idő lenne, hogy a szagok eljussanak hozzánk. Valóságban apró szellők, nehezebb molekulák, molekulák közötti taszítóerők, áramlatok gyorsítják a folyamatot. 
\newpage
\section*{2. feladat}
\subsection*{A feladat szövege}
Írjuk fel az évfolyam évről-évre változó létszámát meghatározó master egyenletet. Gondolkozzunk el azon, hogy mi határozza meg az átmeneti rátákat!

\subsection*{A feladat megoldása}
A master egyenlet általános formáját felírhatjuk:
$$\frac{dP_n(t)}{dt} = -\sum_n\sum_{n'}W_{n'n}P_n(t)+ \sum_{n'}\sum_{n}W_{nn'}P_{n'}(t)$$
Az évfolyamot tekinthetjük egy populációnak, ami egy "születési-halálozási" folyamatban van benne.  A születő egyedeket tekinthetjük a csatlakozó diákoknak, akik "visszacsúsznak" az alattuk lévő évfolyamba vagy Erasmusos cserediákként tagja lesz egy kis ideig a populációnak. Az elhalálozó egyedek pedig akik kikerülnek a rendszerből, otthagyják az egyetemet vagy lejjebb csúsznak. Az átmeneti ráták időfüggetlenek, és a következő módon írhatóak\cite{book}:
$$W_{nn'} = t^+_{n'}\delta_{n,n'+1}+t^-_{n'}\delta_{n,n'-1},$$
ahol $\delta$ a Kronecker-delta szimbólum, $t^\pm_n$ pedig az $n \to n\pm1$ átmenet valószínűsége egységidő alatt. Az általános master egyenletből az átmeneti rátákkal a következő egyenletet kapjuk:
$$\partial_tP_n(t) = t^+_{n-1}P_{n-1}(t)+ t^-_{n+1}P_{n+1}(t) - [t^+_n + t^-_n]P_n(t)$$
Az egyenletnek nincs általános megoldási módszere, stacionárius esetben viszont van. Itt az átmeneti ráták nem annyira véletlen folyamatok, inkább emeberi döntésektől függenek.
\newpage
\section*{3. feladat}
\subsection*{A feladat szövege}
Egy $m$ tömeű részecske $a$ rácsállandójú, egydimenziós rácson $\tau$ időközönként valamelyik
szomszédos rácspontba ugrik. A részecske az origóhoz van kötve egy rugalmas, tömeg nélküli gumiszállal, amelynek rugóállandója $k$, s a környezet hőmérséklete $T$.\\
(a) Írjuk fel a részecske stochasztikus mozgását leíıró master egyenletet!\\
(b) Használjuk a részletes egyensúly elvét konkrét, egyensúlyhoz vezető átmeneti ráták meghatározására!

\subsection*{A feladat megoldása}
A részecske energiája a az $n.$ rácspontban:
$$E_n = E_0+\frac{1}{2}k(an)^2,$$
ahol $E_0$ egy konstans alapállapoti energia. A részecske $n$ állapotból $n\pm 1$ állapotba ugorhat. Az $n.$ állapotban tartózkodás valószínűségét egyensúlyi állapotban felírhatjuk az órán használt módon:
$$P^{(e)}_n = \frac{e^{-\frac{E_n}{k_B T}}}{Z},$$
$Z$ az állapotösszeget jelöli.  A Master egyenlet:
$$\frac{dP_n(t)}{dt }= -W_{n+1,n}P_n(t)-W_{n-1,n}P_{n}(t) + W_{n, n+1}P_{n+1}(t)+W_{n, n-1}P_{n-1}(t)$$
A részletes egyensúly elvét  használva az átmeneti rátákat így számolhatjuk:
$$\frac{W_{n+1, n}}{W_{n, n+1}} = \frac{P_{n+1}}{P_n} = \frac{e^{-\beta E_{n+1}}}{e^{-\beta E_n}} = e^{\beta (E_n-E_{n+1})} = e^{\beta\frac{1}{2}ka^2(1-2n)}$$
$$\frac{W_{n-1,n}}{W_{n, n-1}} = \frac{P_{n-1}}{P_n} = e^{\beta \frac{1}{2}ka^2(2n-1)}$$
A fenti egyenletekben $\beta = k_B T$. Adjunk meg egy olyan feltételt, hogy a részecske ugrási rátája $\frac{1}{\tau}$ legyen, ha az origó felé ugrik, és $\frac{1}{\tau}e^{\beta\frac{1}{2}ka^2(1-2|n|)}$, ha az origótól elfele ugrik. Az abszolútéték az $n$-re a negatív $n$ értékek miatt kell.
 





\begin{thebibliography}{9}

\bibitem{diam}
\texttt{https://www.researchgate.net/post/what\_{}is\_{}the\_{}size\_{}of\_{}water\_{}moleculeH2O}
\bibitem{conc}
\texttt{https://www.quora.com/What-is-the-molar-concentration-of-water-in-1\\-liter-of-pure-water}
\bibitem{book}
 Professor Dr. Crispin W. Gardiner \\Handbook of Stochastic Methods for Physics, Chemistry and the Natural Sciences. Berlin, Heidelberg (1983)


\end{thebibliography}
\end{document}