\documentclass[12pt]{article}
 \usepackage[margin=1in]{geometry} 
\usepackage{amsmath}
\usepackage[utf8]{inputenc}
%\usepackage[ utf8 ]{inputenc}

\newcommand{\angstrom}{\text{\normalfont\AA}}
\usepackage[magyar]{babel}
\newcommand{\N}{\mathbb{N}}
\newcommand{\Z}{\mathbb{Z}}
\usepackage{t1enc}
\usepackage{placeins}
\usepackage{caption}
\usepackage{textcomp}
\usepackage[utf8]{inputenc}
\usepackage[T1]{fontenc}
\title{Véletlen fizikai folyamatok, második házi feladat}
\author{Horváth Bendegúz}


\begin{document}
\maketitle
\section*{1. feladat}
\subsection*{A feladat szövege}
Egy szobában T=200$\text{textdegree{C}}$ hőmérsékletű ideális gáznak tekinthető levegő van. Számítsuk
ki mennyi idő alatt jut el egy vízmolekula a szoba egyik végéből a másikba tisztán diffúziv mozgással.\\
Segítség:
A kinetikus elmélet a gázok diffúziós együtthatójára a következő kifejezést adja (és az eredményt érthetjük is a Brown mozgásról tanultak alapján):
$$ D = \frac{1}{3}l\overline{v}\hspace{1 cm}\Big (= \frac{l^2}{3l/\overline{v}}\approx\frac{(\Delta x)^2}{2\tau}\Big )$$
     ahol l a molekulák szabad úthossza, v pedig átlagos sebességük. A szabad úthosszt megbecsülhetjük a $l = 1/(n\pi d^2)$ kifejezésből, ahol n a molekulák koncentrációja és d a molekulák átmérője. Az átlagos sebességet pedig az ekvipartíció tételéből számolhatjuk.
A valóságban a szagok sokkal gyorsabban terjednek egy szobában, annak ellenére, hogy a megfelelő molekulák lényegesen nagyobbak és súlyosabbak a vízmolekulánál. Értjük ezt?


\subsection*{A feladat megoldása}


\newpage
\section*{2. feladat}
\subsection*{A feladat szövege}
Írjuk fel az évfolyam évről-évre változó létszámát meghatározó master egyenletet. Gon-
dolkozzunk el azon, hogy mi határozza meg az átmeneti rátákat!

\subsection*{A feladat megoldása}

\newpage
\section*{3. feladat}
\subsection*{A feladat szövege}
Egy $m$ tömeű részecske $a$ rácsállandójú, egydimenziós rácson $\tau$ időközönként valamelyik
szomszédos rácspontba ugrik. A részecske az origóhoz van kötve egy rugalmas, tömeg nélküli gumiszállal, amelynek rugóállandója $k$, s a környezet hőmérséklete $T$.\\
(a) Írjuk fel a részecske stochasztikus mozgását leíıró master egyenletet!\\
(b) Használjuk a részletes egyensúly elvét konkrét, egyensúlyhoz vezető átmeneti ráták meghatározására!

\subsection*{A feladat megoldása}

\end{document}